\documentclass[a4paper,12pt]{article}
\usepackage{xeCJK}
\usepackage{fontspec}
\usepackage{listings}
\usepackage{indentfirst}
\setCJKmainfont{SimSun}
\setCJKmonofont{SimSun}
\setmainfont{Times New Roman}
\title{Win32 GUI}
\author{郝鑫}
\newfontfamily{\ttconsolas}{Consolas}
\begin{document}
\maketitle
\tableofcontents
\section{wWinMain函数}
支持UNICODE的wWinMain函数。
\lstset{language=C++,escapechar=`,basicstyle=\ttconsolas,frame=single,breaklines=true}
\begin{lstlisting}
int WINAPI wWinMain(
    HINSTANCE hInstance,
    HINSTANCE hPrevInstance,
    PWSTR pCmdLine,
    int nCmdShow
    );
\end{lstlisting}
WINAPI就是\_\_stdcall
\footnote{
\_\_cdecl:这是C/C++函数默认的调用规范,参数从右向左依次传递,压入堆栈,由调用函数负责堆栈的清退。这种方式适用于传递个数可变的参数给被调用函数,因为只有调用函数才知道它传递了多少个参数给被调函数。如printf函数。
\_\_stdcall:参数从右向左依次传递,并压入堆栈,由被调用函数清退堆栈。该规范生成的函数代码比\_\_cdecl更小,但当函数有可变个数参数,自动转化为\_\_cdecl调用规范。
}。
hInstance是标识本程序的“实例句柄”,hPrevInstance在32位Windows中不再使用,总是NULL。pCmdLine是UNICODE的命令行参数。nShowCmd表示程序最初如何显示(正常,最小/大化《Windows程序设计》P52)。
\section{创建window}
\begin{lstlisting}
const wchar_t CLASS_NAME[] = L"Mages";
WNDCLASS wc = {};
wc.lpfnWndProc = WindowProc;
wc.hInstance = hInstance;
wc.lpszClassName = CLASS_NAME;
RegisterClass(&wc);
HWND hwnd = CreateWindowEx(
    0,// window style
    CLASS_NAME,
    L"Mages",
    WS_OVERLAPPEDWINDOW,//window style
    CW_USEDEFAULT,
    CW_USEDEFAULT,
    CW_USEDEFAULT,
    CW_USEDEFAULT,//size,position
    NULL, // parent window
    NULL, // menu
    hInstance,
    NULL // additional data
    );
if (hwnd == NULL)
{
    return 0;
}
ShowWindow(hwnd, nCmdShow);
MSG msg = {};
while (GetMessage(&msg,NULL,0,0))
{
    TranslateMessage(&msg);
    DispatchMessage(&msg);
}
\end{lstlisting}
WS\_OVERLAPPEDWINDOW是多个标记位的复合,包含标题栏、左边的系统菜单按钮、尺寸调整边框和右上的三个按钮。TranslateMessage转换某些键盘消息。
\section{D3D初始化}
为什么双缓冲能解决闪屏问题?因为如果按像素画下来,上方的像素和下方的像素不是以同时的时间出现或消失,就会看起来感觉是闪烁。


\end{document}